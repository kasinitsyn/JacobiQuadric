\documentclass[12pt]{article}
\usepackage[utf8]{inputenc}
\usepackage[russian]{babel}
\usepackage{amsmath}
\usepackage{amssymb}
\usepackage{graphicx}
\usepackage{alltt}
\usepackage{algorithm}
\usepackage{algorithmic}
\usepackage[colorlinks,urlcolor=blue]{hyperref}
\usepackage{listings} 
\usepackage{color}



\begin{document}
\begin{titlepage}                                                         
\newpage                                                                        
\begin{center}                                                        
ПРАВИТЕЛЬСТВО РОССИЙСКОЙ ФЕДЕРАЦИИ\linebreak
Федеральное государственное автономное образовательное\linebreak
учреждение высшего образования\\
\vspace{2em}
Национальный исследовательский университет\linebreak
«Высшая школа экономики»\\
\vspace{2em}
Московский институт электроники и математики им. А. Н. Тихонова\\
\vspace{2em}
Кафедра компьютерной безопасности\\
                                                                                                                                                                                                                                                        
\vspace{5em}      
                                                                              
\Large Отчёт\linebreak
по курсовой работе по дисциплине\linebreak
“Программирование алгоритмов защиты информации”\linebreak             
                                                                                                                                         
\vspace{6em}                                                          
\end{center}                                                                                                                            

\begin{flushright}
Выполнил студент группы СКБ 172\linebreak
Синицын Константин Алексеевич\\
\end{flushright}
\vspace{\fill}       

\begin{center}              
Москва 2020
\end{center}

\end{titlepage}     


\newpage
{
\hypersetup{linkcolor=black}
\tableofcontents
}
\newpage


\section{Введение}
Данный отчет является результатом выполнения работы по созданию программной реализации алгоритма вычисления кратной точки на эллиптической кривой в форме квадрики Якоби. \\
\\*
\textbf{Задание:}
\begin{itemize}
\item Необходимо:
	\begin{itemize}
	\item Построить/выбрать точку $P$ на кривой;
	\item Выбрать случайное значение $k$;
	\item Реализовать алгоритм вычисления кратной точки $Q = [k]P$;
	\item Провести тестирование программы.
	\end{itemize}
\item Для проведения тестирования необходимо:
	\begin{itemize}
	\item Проверить, что результирующая точка $Q$ лежит на кривой;
	\item Проверить, что $[q]P = \mathcal{O}$, где $q$ - порядок группы точек;
	\item Проверить, что $[q + 1]P = P$ и $[q - 1]P = - P$;
	\item Для двух случайных $k_1, k_2$ проверить, что $[k_1]P + [k_2]P = [k_1 + k_2]P$.
	\end{itemize}
\end{itemize}
\newpage


\section{Теоретическая часть}
\subsection{Квадрика Якоби}
Эллиптическая кривая в форме квадрики Якоби имеет следующий вид:
\begin{center}
$Y^2 = eX^4 - 2dX^2Z^2 + Z^4,$
\end{center}
где $e, d$ - некоторые коэффициенты, $(X : Y : Z)$ - точка на данной кривой, заданная в проективных координатах; $e, d, X, Y, Z \in F_p$, где $p$ - простое и $p > 3$. \\
\\*
Кривая в краткой форме Вейерштрасса (афинная форма) имеет следующий вид:
\begin{center}
$y^2 \equiv x^3 - ax + b (mod p),$
\end{center}
где $a, b$ - параметры кривой в краткой форме Вейерштрасса, $(x, y)$ - точка на данной кривой, заданная в афинных координатах; $a, b, x, y \in F_p$ и $4a^3 + 27b^2 \neq 0 (mod p)$.\\
\\*
Для перехода от проективных координат $(X : Y : Z)$ к афинным координатам $(x, y)$ можно воспользоваться следующими формулами: \\
\begin{equation*}
\begin{cases}
x = \frac{X}{Z}
\\
y = \frac{Y}{Z^2}
\end{cases}
\end{equation*}
Чтобы найти параметры кривой в форме квадратики Якоби, необходимо воспользоваться переходами к ней от краткой формы Вейерштрасса:
\begin{equation*}
\begin{cases}
(\theta, 0) \rightarrow (0 : -1 : 1)
\\
(x, y) \rightarrow (2(x - \theta) : (2x + \theta)(x - \theta)^2 - y^2 : y)
\\
\mathcal{O} \rightarrow (0 : 1 : 1)
\end{cases}
\end{equation*}
Здесь $\theta$ - координата точки второго порядка $(\theta, 0)$, принадлежащей кривой в краткой форме Вейерштрасса. Зная значение $\theta$, можно найти значения параметров $e$ и $d$ согласно формулам $e = \frac{-(3\theta^2 + 4a)}{16}$, $d = \frac{3\theta}{4}$.\\
\\*
\textbf{Определение 1.} \textit{Нейтральный элемент - такая точка $\mathcal{O}$, что выполняются следующие свойства:}
\begin{enumerate}
\item $\mathcal{O} + \mathcal{O} = \mathcal{O}$
\item $\mathcal{O} + P = P + \mathcal{O} - P,$ \textit{где} $P$ \textit{- точка на эллимптической кривой}
\end{enumerate}
Для эллиптической кривой в форме квадратики Якоби нейтральный элемент равен $(0 : 1 : 1)$. \\
\\*
\textbf{Определение 2.} \textit{Обратным элементом к точке} $(X : Y : Z)$ \textit{является} $(-X : Y : Z)$. \\
\\*
\textbf{Определение 3.} \textit{Порядком точки} $P$ \textit{называется такое минимальное число q, что} $[q]P = 0,$ \textit{а также выполняется следующее:}
\begin{enumerate}
\item $[q + 1]P = P$
\item $[q - 1]P = - P$
\end{enumerate}

\subsection{Арифметические операции}
Поскольку точки данной кривой принадлежат аддитивной абелевой группе, для них можно определить операции \textit{сложения} двух различных точек и \textit{удвоения} одной точки. Для кривой в форме квадрики Якоби удвоение является операцией сложения точки с самой собой, поэтому можно обойтись только операцией сложения.

\subsubsection{Сложение}
Для кривой в форме квадрики Якоби формулы сложения двух точек $(X_1 : Y_1 : Z_1) + (X_2 : Y_2 : Z_2) = (X_3 : Y_3 : Z_3)$ выглядят следующим образом:
\begin{equation*}
\begin{cases}
X_3 = X_1Z_1Y_2 + Y_1X_2Z_2
\\
Y_3 = (Z_1^2Z_2^2 + eX_1^2X_2^2)(Y_1Y_2 - 2dX_1X_2Z_1Z_2) + 2eX_1X_2Z_1Z_2(X_1^2Z_2^2 + Z_1^2X_2^2)
\\
Z_3 = Z_1^2Z_2^2 - eX_1^2X_2^2
\end{cases}
\end{equation*}
\\*
\underline{Алгоритм сложения двух точек:}\\*
$T_1 \leftarrow X_1; T_2 \leftarrow Y_1; T_3 \leftarrow Z_1; T_4 \leftarrow X_2; T_5 \leftarrow Y_2; T_6 \leftarrow Z_2$ \\
$T_7 \leftarrow T_1 \cdot T_3$ \hfill $(= X_1Z_1)$ \\*
$T_7 \leftarrow T_2 + T_7$ \hfill $(= X_1Z_1 + Y_1)$ \\*
$T_8 \leftarrow T_4 \cdot T_6$ \hfill $(= X_2Z_2)$ \\*
$T_8 \leftarrow T_5 + T_8$ \hfill $(= X_2Z_2 + Y_2)$ \\*
$T_2 \leftarrow T_2 \cdot T_5$ \hfill $(= Y_1Y_2)$ \\*
$T_7 \leftarrow T_7 \cdot T_8$ \hfill $(= X_3 + Y_1Y_2 + X_1X_2Z_1Z_2)$ \\*
$T_7 \leftarrow T_7 - T_2$ \hfill $(= X_3 + X_1X_2Z_1Z_2)$ \\*
$T_5 \leftarrow T_1 \cdot T_4$ \hfill $(= X_1X_2)$ \\*
$T_1 \leftarrow T_1 + T_3$ \hfill $(= X_1 + Z_1)$ \\*
$T_8 \leftarrow T_3 \cdot T_6$ \hfill $(= Z_1Z_2)$ \\*
$T_4 \leftarrow T_4 + T_6$ \hfill $(= X_2 + Z_2)$ \\*
$T_6 \leftarrow T_5 \cdot T_8$ \hfill $(= X_1X_2Z_1Z_2)$ \\*
$T_7 \leftarrow T_7 - T_6$ \hfill $(= X_3)$ \\*
$T_1 \leftarrow T_1 \cdot T_4$ \hfill $(= X_1Z_2 + X_2Z_1 + X_1X_2 + Z_1Z_2)$ \\*
$T_1 \leftarrow T_1 - T_5$ \hfill $(= X_1Z_2 + X_2Z_1 + Z_1Z_2)$ \\*
$T_1 \leftarrow T_1 - T_8$ \hfill $(= X_1Z_2 + X_2Z_1)$ \\*
$T_3 \leftarrow T_1 \cdot T_1$ \hfill $(= X_1^2Z_2^2 + X_2^2Z_1^2 + 2X_1X_2Z_1Z_2)$ \\*
$T_6 \leftarrow T_6 + T_6$ \hfill $(= 2X_1X_2Z_1Z_2)$ \\*
$T_3 \leftarrow T_3 - T_6$ \hfill $(= X_1^2Z_2^2 + X_2^2Z_1^2)$ \\*
$T_4 \leftarrow e \cdot T_6$ \hfill $(= 2eX_1X_2Z_1Z_2)$ \\*
$T_3 \leftarrow T_3 \cdot T_4$ \hfill $(= 2eX_1X_2Z_1Z_2(X_1^2Z_2^2 + X_2^2Z_1^2))$ \\*
$T_4 \leftarrow d \cdot T_6$ \hfill $(= 2dX_1X_2Z_1Z_2)$ \\*
$T_2 \leftarrow T_2 - T_4$ \hfill $(= Y_1Y_2 - 2dX_1X_2Z_1Z_2)$ \\*
$T_4 \leftarrow T_8 \cdot T_8$ \hfill $(= Z_1^2Z_2^2)$ \\*
$T_8 \leftarrow T_5 \cdot T_5$ \hfill $(= X_1^2X_2^2)$ \\*
$T_8 \leftarrow e \cdot T_8$ \hfill $(= eX_1^2X_2^2)$ \\*
$T_5 \leftarrow T_4 + T_8$ \hfill $(= Z_1^2Z_2^2 + eX_1^2X_2^2)$ \\*
$T_2 \leftarrow T_2 \cdot T_5$ \hfill $(= (Z_1^2Z_2^2 + eX_1^2X_2^2)(Y_1Y_2 - 2dX_1X_2Z_1Z_2))$ \\*
$T_2 \leftarrow T_2 + T_3$ \hfill $(= Y_3)$ \\*
$T_5 \leftarrow T_4 - T_8$ \hfill $(= Z_3)$ \\*
$X_3 \leftarrow T_7; Y_3 \leftarrow T_2; Z_3 \leftarrow T_5$

\subsubsection{Нахождение кратной точки}
\textbf{Определение 4.} \textit{Пусть}$P$\textit{ - точка на кривой, тогда} $[k]P = \underbrace{P + P + \dots + P}_{k раз}$ \textit{- кратная точка}, $k \in \mathbb{Z}, 0 \le k < q$. \\
\\*
Самый эффективный способ вычисления кратной точки это алгоритм "Лесенка Монтгомери".
\begin{algorithm}
\caption{Лесенка Монтгомери}
\begin{algorithmic}[1]
\STATE получить двоичное представление $k = (k_{n - 1}, \dots, k_0) = \sum^{n - 1}_{n = 0}k_i2^i$
\STATE определить $Q = \mathcal{O}, R = P$
\FOR{$i \leftarrow n - 1$ to 0}
	\IF{$k_i = 0$}
		\STATE вычислить $R = R + Q$ и $Q = [2]Q$;
	\ENDIF
	\IF{$k_i = 1$}
		\STATE вычислить $Q = Q + R$ и $R = [2]R$;
	\ENDIF
\ENDFOR
\STATE определить в качестве результата $Q$
\end{algorithmic}
\end{algorithm}
\newpage


\section{Работа с библиотекой libtommath. Описание функций.}
Основным типом данных в данной библиотеке является тип \texttt{mp\_int}, предназначенный для хранения больших целых чисел. При написании реализации были использованы следующие функции библиотеки:\\
\\*
\textcolor{red}{\texttt{int}} \textcolor{blue}{\texttt{mp\_init}}\texttt{(mp\_int * a)} - инициализирует структуру \texttt{mp\_int а} и выделяет память для хранения большого числа.\\
\\*
\textcolor{red}{\texttt{int}} \textcolor{blue}{\texttt{mp\_init\_multi}}\texttt{(mp\_int * mp, ..., NULL)} - инициализирует несколько структур \texttt{mp\_int} и выделяет память для хранения больших чисел. \\
\\*
\textcolor{red}{\texttt{int}} \textcolor{blue}{\texttt{mp\_init\_set}}\texttt{(mp\_int * a, mp\_digit b)} - выделяет память и инициализирует \texttt{mp\_int a} однозначным числом \texttt{mp\_digit b}. В качестве аргумента также можно передавать небольшие числа (не более \texttt{short int}).\\
\\*
\textcolor{red}{\texttt{int}} \textcolor{blue}{\texttt{mp\_init\_copy}}\texttt{(mp\_int * a, mp\_int * b)} - выделяет память и инициализирует \texttt{mp\_int a} копией значения \texttt{mp\_int b}.\\
\\*
\textcolor{red}{\texttt{void}} \textcolor{blue}{\texttt{mp\_clear}}\texttt{(mp\_int * a)} - освобождает память, использующуяся для хранения структуры \texttt{mp\_int a}.\\
\\*
\textcolor{red}{\texttt{int}} \textcolor{blue}{\texttt{mp\_clear\_multi}}\texttt{(mp\_int *mp, \dots, NULL)} - освобождает память, использующиеся для хранения каждой из структур \texttt{mp\_int}.\\
\\*
\textcolor{red}{\texttt{int}} \textcolor{blue}{\texttt{mp\_cmp}}\texttt{(mp\_int * a, mp\_int * b)} - производит знаковое сравнение чисел \texttt{mp\_int a} и \texttt{mp\_int b}. Возвращает \texttt{MP\_EQ} в случае равенства, \texttt{MP\_GT}, если a > b и \texttt{MP\_LT}, если a < b. \\
\\*
\textcolor{red}{\texttt{int}} \textcolor{blue}{\texttt{mp\_add}}\texttt{(mp\_int * a, mp\_int * b, mp\_int * c)} - записывает в \texttt{c} результат суммы \texttt{a + b}.\\
\\*
\textcolor{red}{\texttt{int}} \textcolor{blue}{\texttt{mp\_sub}}\texttt{(mp\_int * a, mp\_int * b, mp\_int * c)} - записывает в \texttt{с} результат разности \texttt{a - b}.\\
\\*
\textcolor{red}{\texttt{int}} \textcolor{blue}{\texttt{mp\_neg}}\texttt{(mp\_int * a, mp\_int * b)} - записывает в \texttt{b} значение \texttt{a} с противоположным знаком (\texttt{-a}).\\
\\*
\textcolor{red}{\texttt{int}} \textcolor{blue}{\texttt{mp\_div}}\texttt{(mp\_int * a, mp\_int * b, mp\_int * c, mp\_int * d)} - делит с остатком \texttt{a} на \texttt{b} и записывает частное в \texttt{c}, остаток в \texttt{d} (\texttt{a = bc + d)}). В случае ненадобности \texttt{mp\_int * c} или \texttt{mp\_int * d} может быть заменено на \texttt{NULL} (для получения, соответственно, только частного или только остатка). \\
\\*
\textcolor{red}{\texttt{int}} \textcolor{blue}{\texttt{mp\_mul}}\texttt{(mp\_int * a, mp\_int * b, mp\_int * c)} - помещает результат умножения \texttt{a} на \texttt{b} в \texttt{c}.\\
\\*
\textcolor{red}{\texttt{int}} \textcolor{blue}{\texttt{mp\_exptmod}}\texttt{(mp\_int * G, mp\_int * X, mp\_int * P, mp\_int * Y)} - возводит \texttt{G} в степень \texttt{Х} и помещает результат по модулю \texttt{Р} в \texttt{Y} (\texttt{$Y \equiv G X (mod P)$}).\\
\\*
\textcolor{red}{\texttt{int}} \textcolor{blue}{\texttt{mp\_to\_radix}}\texttt{(const mp\_int * a, char * str, size\_t maxlen, size\_t * written, int radix)} - помещает \texttt{mp\_int a} в строковом виде в \texttt{char str}. В \texttt{radix} указывается основание системы счисления (от 2 до 64), в \texttt{size\_t maxlen} - максимальный размер, который может занять число в \texttt{char str}, в \texttt{size\_t written} помещается размер реально записанного в \texttt{char str} числа.\\
\\*
\textcolor{red}{\texttt{int}} \textcolor{blue}{\texttt{mp\_radix\_size}}\texttt{(mp\_int * a, int radix, int *size)} - помещает в \texttt{size} целое число – размер, необходимый для помещения \texttt{mp\_int а} в char. В \texttt{radix} указывается основание системы счисления (от 2 до 64). \\
\\*
\textcolor{red}{\texttt{int}} \textcolor{blue}{\texttt{mp\_read\_radix}}\texttt{(mp\_int * a, char *str, int radix)} - считывает число в строковом виде из \texttt{char str} и помещает в \texttt{mp\_int a}. В \texttt{radix} указывается основание системы счисления (от 2 до 64).\\
\\*
\textcolor{red}{\texttt{int}} \textcolor{blue}{\texttt{mp\_prime\_random}}\texttt{(mp\_int * a, int t, int size, int bbs, ltm\_prime\_callback cb, void * dat)} - генерирует случайное простое число, превышающее $256^{size}$ и помещает его в \texttt{mp\_int a}. При этом число должно пройти \texttt{t} тестов. В аргументах \texttt{int bbs, ltm\_prime\_callback cb, void * dat} указываются дополнительные флаги (не используется). Данная функция используется в программе для генерации случайного числа, поскольку библиотека Libtommath не содержит функций для генерации случайных чисел.\\ 
\\*


\section{Описание основных функций и структур}
Полные исходные коды можно найти в репозитории по ссылке \url{https://github.com/kasinitsyn/JacobiQuadric}. Ниже представлено описание основных функций и структур.
\subsection{Основные структуры}
Структура \texttt{Point} - точка в проективных координатах $(X, Y, Z)$. Каждая из координат имеет тип \texttt{mp\_int}.
\lstset{language=C}
\begin{lstlisting} 
struct Point
{
    mp_int X;
    mp_int Y;
    mp_int Z;
};
\end{lstlisting}
Структура \texttt{Parameters} содержит параметры, определенные стандартом (подробное описание параметров см. Исходные данные)
\lstset{language=C}
\begin{lstlisting} 
struct Parameters
{
    mp_int p;
    mp_int q;

    mp_int a;
    mp_int b;

    mp_int x_base;
    mp_int y_base;

    mp_int theta;
};
\end{lstlisting}
Стуктура \texttt{JacobiQuadric} содержит параметры квадрики Якоби: $p$ - характеристика поля, $e, d$ - параметры кривой, $X, Y, Z$ - координаты порождающего элемента.
\lstset{language=C}
\begin{lstlisting} 
struct JacobiQuadric
{
    mp_int p;
    mp_int e;
    mp_int d;

    mp_int X;
    mp_int Y;
    mp_int Z;
};
\end{lstlisting} 
\subsection{Основные функции}
\subsubsection{Функции, обеспечивающие арифметику по модулю числа}
В виду отсутствия в библиотеке Libtommath необходимых для математики эллиптических кривых функций, осуществляющих арифметику по модулю числа, в ходе выполнения данной работы были написаны соответствующие функции. Функции стилистически оформлены как функции библиотеки.\\
\\*
\textcolor{red}{\texttt{void}} \textcolor{blue}{\texttt{mp\_add\_mod}}\texttt{(mp\_int * a, mp\_int * b, mp\_int * c, mp\_int * mod)} - записывает в \texttt{c} результат суммы \texttt{a + b} по модулю числа \texttt{p}.\\
\\*
\textcolor{red}{\texttt{void}} \textcolor{blue}{\texttt{mp\_sub\_mod}}\texttt{(mp\_int * a, mp\_int * b, mp\_int * c, mp\_int * mod)} - записывает в \texttt{c} результат разности \texttt{a - b} по модулю числа \texttt{p}.\\
\\*
\textcolor{red}{\texttt{void}} \textcolor{blue}{\texttt{mp\_mul\_mod}}\texttt{(mp\_int * a, mp\_int * b, mp\_int * c, mp\_int * mod)} - записывает в \texttt{c} результат произведения \texttt{a * b} по модулю числа \texttt{p}.\\
\\*
\textcolor{red}{\texttt{void}} \textcolor{blue}{\texttt{mp\_neg\_mod}}\texttt{(mp\_int * a, mp\_int * b, mp\_int * mod)} - записывает в \texttt{b} значение \texttt{a} с противоположным знаком (\texttt{-a}) по по модулю числа \texttt{p}.\\

\subsubsection{Функции инициализации основных структур}
\textcolor{red}{\texttt{void}} \textcolor{blue}{\texttt{InitPoint}}\texttt{(struct Point * P, mp\_int * x, mp\_int * y, mp\_int * z)} - инициализация точки \texttt{P} с проективными координатами \texttt{x, y, z}.\\
\\*
\textcolor{red}{\texttt{void}} \textcolor{blue}{\texttt{InitParameters}}\texttt{(struct Parameters * Param, mp\_int * p, mp\_int * q, mp\_int * a, mp\_int * b, mp\_int * x\_base, mp\_int * y\_base, mp\_int * theta)} - инициализация структуры \texttt{Param} с параметрами.\\
\\*
\textcolor{red}{\texttt{void}} \textcolor{blue}{\texttt{InitJacobiQuadric}}\texttt{(struct JacobiQuadric * JQ, struct Parameters * Param)} - инициализация структуры \texttt{JQ} с параметрами кривой.\\ 

\subsubsection{Функции освобождения памяти, использующейся для хранения основных структур}
\textcolor{red}{\texttt{void}} \textcolor{blue}{\texttt{ClearPoint}}\texttt{(struct Point * P)} - освобождение памяти, использовавшейся для хранения точки в структуре \texttt{P}.\\
\\*
\textcolor{red}{\texttt{void}} \textcolor{blue}{\texttt{ClearParameters}}\texttt{(struct Parameters * Param)} - освобождение памяти, использовавшейся для хранения параметров в струкрутре \texttt{Param}.\\
\\*
\textcolor{red}{\texttt{void}} \textcolor{blue}{\texttt{ClearJacobiQuadric}}\texttt{(struct JacobiQuadric * JQ)} - освобождение памяти, использовавшейся для хранения параметров кривой Якоби в структуре \texttt{JQ}.\\

\subsubsection{Функции для вывода координат точки на экран}
\textcolor{red}{\texttt{void}} \textcolor{blue}{\texttt{PrintPoint}}\texttt{(struct Point * P)} - вывести значения проективных координат точки \texttt{P} на экран.\\
\\*
\textcolor{red}{\texttt{void}} \textcolor{blue}{\texttt{PrintPointAffine}}\texttt{(struct Point * P, struct JacobiQuadric * JQ)} - вывести значения афинных координат точки \texttt{P} на экран. Структура \texttt{JQ} - данная кривая.\\

\subsubsection{Основные функции}
\textcolor{red}{\texttt{bool}} \textcolor{blue}{\texttt{IsPointOnCurve}}\texttt{(struct Point * P, struct JacobiQuadric * JQ)} - проверка находится ли данная точка \texttt{P} на данной кривой \texttt{JQ}. Возвращает \texttt{true}, если точка находится на кривой и \texttt{false} в противном случае.\\
\\*
\textcolor{red}{\texttt{void}} \textcolor{blue}{\texttt{Addition}}\texttt{(struct Point * P1, struct Point * P2, struct Point * P3, struct JacobiQuadric * JQ)} - сложение двух точек \texttt{P1 + P2}, результат записывается в третью точку \texttt{P3}. Структура \texttt{JQ} - данная кривая.\\
\\*
\textcolor{red}{\texttt{bool}} \textcolor{blue}{\texttt{ArePointsEqual}}\texttt{(struct Point * P1, struct Point * P2, struct JacobiQuadric * JQ)} - проверка равенства двух точек \texttt{P1} и \texttt{P2} на кривой \texttt{JQ}.\\
\\*
\textcolor{red}{\texttt{void}} \textcolor{blue}{\texttt{MontgomeryLadder}}\texttt{(struct Point * P, mp\_int * k, struct Point * Q, struct JacobiQuadric * JQ)} - реализация алгоритма "лесенка Монтгомери", где \texttt{P} - точка на кривой, \texttt{k} - степень точки, \texttt{Q} - результирующая точка $Q = [k]P$, \texttt{JQ} - данная кривая.\\
\\*


\section{Тестирование}
\subsection{Установка и запуск реализации}
Для установки библиотеки Libtommath следует воспользоваться официальной инструкцией по установке, приведенной в документации к библиотеке \url{https://github.com/libtom/libtommath}. \\
\\*
Для установки и запуска реализации следует выполнить следующие команды в командной строке из папки с проектом:
\begin{enumerate}
\item cmake CMakeLists.txt
\item make
\item ./JacodiQuadric
\end{enumerate}
\subsection{Исходные данные}
Параметры для проверки работоспособности реализации и правильности ее выполнения были взяты из документа \textit{«Рекомендации по стандартизации. Параметры эллиптических кривых для криптографических алгоритмов и протоколов. Р 50.1.114 – 2016»}. Для проверки из предложенных наборов параметров был взят набор параметров \textbf{id-tc26-gost-3410-2012-256-paramSetA}, содержащий следующие значения:\\
\\*
p = $115792089237316195423570985008687907853269984665640564039457584007913129639319_{10}$
a = $87789765485885808793369751294406841171614589925193456909855962166505018127157_{10}$
b = $18713751737015403763890503457318596560459867796169830279162511461744901002515_{10}$
q = $28948022309329048855892746252171976963338560298092253442512153408785530358887_{10}$
x = $65987350182584560790308640619586834712105545126269759365406768962453298326056_{10}$
y = $22855189202984962870421402504110399293152235382908105741749987405721320435292_{10}$\\
\\*
Где:\\*
p - характеристика простого поля, над которым определятся эллиптическая кривая; \\*
a, b - параметры эллиптической кривой в форме Вейерштрасса (параметр $b$ в реализации не используется); \\*
q - порядок циклической подгруппы группы точек эллиптической кривой; \\*
x, y - координаты порождающего элемента в краткой форме Вейерштрасса (в реализации обозначаются как x\_base, y\_base).\\
\\*
В документе значения параметров приведены в шестнадцатиричной системе счисления. Для перевода параметров в десятичную систему использовалась программа Wolfram Mathematica. Также, с помощью данной  программы были вычислена х-координата точки $(\theta, 0)$ второго порядка в форме Вейерштрасса (значение $\theta$): \\
\\* 
$\theta = 454069018412434321972378083527459607666454479745512801572100703902391945898_{10}$\\
\\*
Ниже приведен фрагмент кода Wolfram Mathematica и результат выполнения вычислений:
\begin{center}
\includegraphics [width=1.0\textwidth]{pic1.png}\\
\includegraphics [width=0.85\textwidth]{pic2.png}\\
\end{center}



\subsection{Результаты}
\subsubsection{Проверка на утечки памяти}
Вначале была осуществлена проверка утечек памяти с помощью средства поиска ошибок работы с памятью Valgrind. \\
Команда: valgrind --leak-check=full ./JacobiQuadric\\*
Вывод:\\*
==1756==  HEAP SUMMARY:\\*
==1756==     in use at exit: 0 bytes in 0 blocks\\*
==1756==   total heap usage: 489,445 allocs, 489,445 frees, 55,656,115 bytes allocated\\*
==1756== \\*
==1756== All heap blocks were freed -- no leaks are possible\\*
==1756== \\*
==1756== For counts of detected and suppressed errors, rerun with: -v\\*
==1756== ERROR SUMMARY: 0 errors from 0 contexts (suppressed: 0 from 0)\\
\subsubsection{Тестирование реализации}
Затем было проведено тестирование самой реализации:\\*
Команда: ./JacobiQuadric\\*
Вывод:\\*
Посчитанные параметры квадрики Якоби:\\*
d = 58236596382467423453264776066989548632384833192629416620907867531883358779083\\*
e = 21881292613901449512659201470451780075363042554712173057987834765447108787084\\*
X\_base = 15274473091028057513101540063430842355608196627407929088211752509188683120997\\*
Y\_base = 70639478069546534592066422814913955506998300889114271757947051176576672450210\\*
Z\_base = 22855189202984962870421402504110399293152235382908105741749987405721320435292\\
\\*
ТЕСТ 1: ПРОВЕРКА ПРИНАДЛЕЖНОСТИ НЕЙТРАЛЬНОГО ЭЛЕМЕНТА\\*
Точка в проективных координатах:\\*
X = 0\\*
Y = 1\\*
Z = 1\\*
Точка в афинных координатах:\\*
x = 0\\*
y = 1\\*
Нейтральный элемент Е принадлежит кривой\\
\\*
ТЕСТ 2: ПОРОЖДАЮЩИЙ ЭЛЕМЕНТ В АФИННЫХ КООРДИНАТАХ\\*
Точка в афинных координатах:\\*
x = 26\\*
y = 32588803023257230788452318859724590706198019287541469357859214741485052675122\\*
Порождающий элемент P\_base принадлежит кривой\\
\\*
ТECT 3: E + P\_base = P\_base?\\*
Точка в проективных координатах:\\*
X = 46006328807261240983066223367402584890441940889817085927906907131943988570069\\*
Y = 55579995469928774719018320141672621810816086559469103791209555426079733213871\\*
Z = 50758435016066899640090271479344983510631222008148588090843858893038015946253\\*
Точка в афинных координатах:\\*
x = 26\\*
y = 32588803023257230788452318859724590706198019287541469357859214741485052675122\\*
Точка Е + P\_base принадлежит кривой\\*
Точки E+P\_base и P\_base равны\\
\\*
ТECT 4: Принадлежит ли точка P2 = (5 : 1 : 4) кривой\\*
Точка в афинных координатах:\\*
x = 28948022309329048855892746252171976963317496166410141009864396001978282409831\\*
y = 65133050195990359925758679067386948167464366374422817272194891004451135422117\\*
Точка (5 : 1 : 4) не принадлежит кривой\\
\\*
ТECT 5: [q]P = E?\\*
Точка в проективных координатах:\\*
X = 0\\*
Y = 40178936660781546849967672518756049964755243454469907618687078886735047096265\\*
Z = 57742264586188110633053061231704778952292277256767593810644170837071465729572\\*
Точка в афинных координатах:\\*
x = 0\\*
y = 1\\*
Точки [q]P и E равны\\
\\*
ТECT 6: [q+1]P = P и [q-1] = -P\\*
Точка в афинных координатах:\\*
x = 115792089237316195423570985008687907853269984665640564039457584007913129639293\\*
y = 32588803023257230788452318859724590706198019287541469357859214741485052675122\\*
Точка в афинных координатах:\\*
x = 115792089237316195423570985008687907853269984665640564039457584007913129639293\\*
y = 32588803023257230788452318859724590706198019287541469357859214741485052675122\\*
Точки [q - 1]P и -P равны\\*
Точка в афинных координатах:\\*
x = 26\\*
y = 32588803023257230788452318859724590706198019287541469357859214741485052675122\\*
Точка в афинных координатах:\\*
x = 26\\*
y = 32588803023257230788452318859724590706198019287541469357859214741485052675122\\*
Точки [q + 1]P и P равны\\
\\*
ТECT 7: Вычисление [k]P при k = 100\\*
Точка в афинных координатах:\\*
x = 46114831014247229923266331647927557586696495636126505757008735063481431609683\\*
y = 38376220474406473655225685664497454497247526062573712862044892681609942213050\\*
Точка [k]P принадлежит кривой\\
\\*
ТЕСТ 8: Вычисление [k]P для случайного k из диапазона [0, q)\\*
k = 991954433999604731829632709224396598341591234772024487906631\\*
Точка в афинных координатах:\\*
x = 50779116323969119300621785808242934425388155432437577476919529444328576423118\\*
y = 94020197051731514972631394841409410785510879144286959132168853193003725895704\\*
Точка [k]P принадлежит кривой\\
\\*
ТЕСТ 9: [k1]P + [k2]P = [k1 + k2]P?\\*
k1 = 1084845348725810821418535502021\\*
k2 = 795405475617922960716810407137\\*
k1 + k2 = 1880250824343733782135345909158\\*
Точка в афинных координатах:\\*
x = 36783066602330481256214373320726812578572207207168637666900660686517300314330\\*
y = 52106396355070439400592651537488559251130145451034852674912273346313496501149\\*
Точка [k1]P принадлежит кривой\\*
Точка в афинных координатах:\\*
x = 23653286548373740116138831789119419465516319104618009133532289868355943583259\\*
y = 6521473322108346065594065622514635457973368003972073332546242861921339483508\\*
Точка [k2]P принадлежит кривой\\*
Точка в афинных координатах:\\*
x = 100174933671734223955453094649162785325397815042489168097357339866005748107089\\*
y = 84966962613761404393860727171805411782744711102320690988699985888828907160639\\*
Точка [k1 + k2]P принадлежит кривой\\*
Точки [k1]P + [k2]P и [k1 + k2]P равны\\
\\*
ВСЕ ТЕСТЫ УСПЕШНО ПРОЙДЕНЫ\\
\\*



\section{Использованная литература}
\begin{enumerate}
\item Нестеренко А. Ю. – Курс лекций «Методы программной реализации
СКЗИ»;
\item O. Billet, M. Joye. – «The Jacobi model of an elliptic curve and side-channel
analysis, proceedings of AAECC-15, Lecture Notes in Computer Science» \url{https://eprint.iacr.org/2002/125.pdf};
\item «Рекомендации по стандартизации. Параметры эллиптических кривых
для криптографических алгоритмов и протоколов. Р 50.1.114 – 2016»;
\item Документация библиотеки libtommath «LibTomMath User Manual» \url{https://github.com/libtom/libtommath}.
\end{enumerate}


\end{document}